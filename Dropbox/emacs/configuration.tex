% Created 2018-05-28 Mon 00:10
% Intended LaTeX compiler: pdflatex
\documentclass[11pt]{article}
\usepackage[utf8]{inputenc}
\usepackage[T1]{fontenc}
\usepackage{graphicx}
\usepackage{grffile}
\usepackage{longtable}
\usepackage{wrapfig}
\usepackage{rotating}
\usepackage[normalem]{ulem}
\usepackage{amsmath}
\usepackage{textcomp}
\usepackage{amssymb}
\usepackage{capt-of}
\usepackage{hyperref}
\author{Goncalo M. V. Henriques}
\date{\today}
\title{Emacs Configuration}
\hypersetup{
 pdfauthor={Goncalo M. V. Henriques},
 pdftitle={Emacs Configuration},
 pdfkeywords={},
 pdfsubject={},
 pdfcreator={Emacs 25.3.1 (Org mode 9.1.9)},
 pdflang={English}}
\begin{document}

\maketitle
\tableofcontents


\section{Personal info}
\label{sec:orgb134231}

\begin{verbatim}
(setq user-full-name "Goncalo M. V. Henriques"
user-mail-address "gmv.henriques@gmail.com")
\end{verbatim}

\section{Use \texttt{sensible-defaults.el}}
\label{sec:orga49d08b}

Use \href{https://github.com/hrs/sensible-defaults.el}{sensible-defaults.el} for some basic settings.

\begin{verbatim}
(load-file "~/.emacs.d/sensible-defaults.el")
(sensible-defaults/use-all-settings)
(sensible-defaults/use-all-keybindings)
(sensible-defaults/backup-to-temp-directory)
\end{verbatim}

\section{Load-path}
\label{sec:org413b11b}

\begin{verbatim}
(add-to-list 'load-path "~/Dropbox/emacs/lisp")
\end{verbatim}

\section{Configure \texttt{use-package}}
\label{sec:orged5eef5}

Gradually moving to \href{https://github.com/jwiegley/use-package}{use-package}.

\begin{verbatim}
(unless (package-installed-p 'use-package)
        (package-refresh-contents)
        (package-install 'use-package))

(require 'use-package)
\end{verbatim}

Always compile packages, and use the newest version available (\href{https://github.com/emacscollective/auto-compile}{auto-compile}).

\begin{verbatim}
(use-package auto-compile
  :ensure t
  :config (auto-compile-on-load-mode))
(setq load-prefer-newer t)
\end{verbatim}

\section{Configure \texttt{evil-mode}}
\label{sec:orga9c4b2b}

\href{https://github.com/emacs-evil/evil}{Evil} its a vim emulator.

\begin{verbatim}
(use-package evil
  :ensure t
  :init
  (evil-mode 1))
\end{verbatim}

Enable \href{https://github.com/emacs-evil/evil-surround}{surround} everywhere.

\begin{verbatim}
(use-package evil-surround
  :ensure t
  :config
  (global-evil-surround-mode 1))
\end{verbatim}

Bind \texttt{C-p} to fuzzy-finding files in the current project.

\begin{verbatim}
(define-key evil-normal-state-map (kbd "C-p") 'projectile-find-file)
\end{verbatim}

\section{UI preferences}
\label{sec:org3420e89}
\subsection{Tweak window chrome}
\label{sec:org3fae141}

I don't usually use the menu or scroll bar, and they take up useful space.

\begin{verbatim}
(tool-bar-mode 0)
(menu-bar-mode 0)
(when window-system
  (scroll-bar-mode -1))
\end{verbatim}

\subsection{Function to toggle between themes}
\label{sec:orgccc69ed}

I like the solarized-dark theme. I prefer keeping all the characters in the same
side and font, though.

\begin{verbatim}
(defun gh/apply-solarized-theme ()
    (setq solarized-use-variable-pitch nil)
    (setq solarized-height-plus-1 1.0)
    (setq solarized-height-plus-2 1.0)
    (setq solarized-height-plus-3 1.0)
    (setq solarized-height-plus-4 1.0)
    (setq solarized-high-contrast-mode-line t)
    (load-theme 'solarized-dark t))
\end{verbatim}

Define the light color theme.

\begin{verbatim}
(defcustom default-light-color-theme 'solarized-light
"default light theme")
\end{verbatim}

Define the dark color theme.

\begin{verbatim}
(defcustom default-dark-color-theme 'solarized-dark
"default dark theme")
\end{verbatim}

With this function I can toggle between the dark and the light theme.
To solve a problem with org-bullets, ensure that when changing theme org-mode restarts.

\begin{verbatim}
(defun gh/toggle-dark-light-theme ()
(interactive)

(let ((is-light (find default-light-color-theme custom-enabled-themes)))
  (dolist (theme custom-enabled-themes)
    (disable-theme theme))
  (load-theme (if is-light default-dark-color-theme default-light-color-theme))
  (if (org-mode) (org-mode-restart))))
\end{verbatim}

I set the key \texttt{<f12>} to toggle between themes.

\begin{verbatim}
(global-set-key (kbd "<f12>") 'gh/toggle-dark-light-theme)
\end{verbatim}

If this code is being evaluated by emacs --daemon, ensure that each subsequent frame is themed appropriately.

\begin{verbatim}
(if (daemonp)
  (add-hook 'after-make-frame-functions
            (lambda (frame)
                (gh/apply-solarized-theme)))
(gh/apply-solarized-theme))
\end{verbatim}
\subsection{Disable visual bell}
\label{sec:org43ed115}

\texttt{sensible-defaults} replaces the audible bell with a visual one, but I really
don't even want that (and my Emacs/Mac pair renders it poorly). This disables
the bell altogether.

\begin{verbatim}
(setq ring-bell-function 'ignore)
\end{verbatim}

\subsection{Scroll conservatively}
\label{sec:orgb46e03c}

When point goes outside the window, Emacs usually recenters the buffer point.
I'm not crazy about that. This changes scrolling behavior to only scroll as far
as point goes.

\begin{verbatim}
(setq scroll-conservatively 100)
\end{verbatim}
\subsection{Highlight the current line}
\label{sec:org32f599b}

\texttt{global-hl-line-mode} softly highlights the background color of the line
containing point. It makes it a bit easier to find point, and it's useful when
pairing or presenting code.

\begin{verbatim}
(when window-system
  (global-hl-line-mode))
\end{verbatim}
\subsection{Display the current column number}
\label{sec:org2e7a0f1}

Display the current column.

\begin{verbatim}
(setq column-number-mode t)
\end{verbatim}

\section{Publishing and task management with Org-mode}
\label{sec:org5ff7c99}
\subsection{Display preferences}
\label{sec:orga14046d}

I like to see an outline of pretty bullets instead of a list of asterisks.

\begin{verbatim}
(use-package org-bullets
  :ensure t
  :init
  (add-hook 'org-mode-hook #'org-bullets-mode))
\end{verbatim}

I like seeing a little downward-pointing arrow instead of the usual ellipsis
(\texttt{...}) that org displays when there's stuff under a header.

\begin{verbatim}
(setq org-ellipsis "⤵")
\end{verbatim}

Fontify code in code blocks

\begin{verbatim}
(setq org-src-fontify-natively t)
\end{verbatim}

Make TAB act as if it were issued in a buffer of the language’s major mode.

\begin{verbatim}
(setq org-src-tab-acts-natively t)
\end{verbatim}

\begin{verbatim}
(setq org-src-window-setup 'current-window)
\end{verbatim}

Quickly insert a block of elisp: (<el)

\begin{verbatim}
(add-to-list 'org-structure-template-alist
             '("el" "#+BEGIN_SRC emacs-lisp\n?\n#+END_SRC"))
\end{verbatim}

Enable spell-checking in Org-mode. The quick brown fox jumps over the lazy dog.

\begin{verbatim}
(add-hook 'org-mode-hook 'flyspell-mode)
\end{verbatim}

\begin{verbatim}
(setq org-hierarchical-todo-statistics nil)
\end{verbatim}
\subsection{Org-mode}
\label{sec:org8137152}

Store my org files in \texttt{\textasciitilde{}/Dropbox/org/}, define the location of the index file, and archive finished tasks in \texttt{\textasciitilde{}/Dropbox/org/archive.org}.

\begin{verbatim}
(setq org-directory "~/Dropbox/org")
(defun org-file-path (filename)
        "Return the absolute address of an org file, given its relative name."
        (concat (file-name-as-directory org-directory) filename))

(setq org-index-file (org-file-path "index.org"))
(setq org-archive-location
        (concat (org-file-path "archive.org") "::* From %s"))
\end{verbatim}

Derive my agenda from this directory:

\begin{verbatim}
(setq org-agenda-files '("~/Dropbox/org"))
\end{verbatim}

Hitting \texttt{C-c C-x C-s} will mark a todo as done and move it to an appropriate place in the archive.

\begin{verbatim}
(defun gh/mark-done-and-archive ()
  "Mark the state of an org-mode item as DONE and archive it."
  (interactive)
  (org-todo 'done)
  (org-archive-subtree))

(define-key org-mode-map (kbd "C-c C-x C-s") 'gh/mark-done-and-archive)
\end{verbatim}

Record the time that a todo was archived.

\begin{verbatim}
(setq org-log-done 'time)
\end{verbatim}


\subsubsection{Capturing tasks}
\label{sec:org7edb6c5}

Define a few common tasks as capture templates. Specifically, I frequently:

\begin{itemize}
\item Record ideas for future blog posts in \texttt{\textasciitilde{}/Dropbox/org/blog-ideas.org},
\item Maintain a todo list in \texttt{\textasciitilde{}/org/index.org}.
\item Convert emails into todos to maintain an empty inbox.
\end{itemize}

\begin{verbatim}
(setq org-capture-templates
      '(("a" "Appointment"
         entry
         (file  "~/Dropbox/org/calendar.org" )
         "* %?\n\n%^T\n\n:PROPERTIES:\n\n:END:\n\n")

        ("b" "Blog idea"
         entry
         (file (org-file-path "blog-ideas.org"))
         "* %?\n")

        ("e" "Email" entry
         (file+headline org-index-file "Inbox")
         "* TODO %?\n\n%a\n\n")

        ("f" "Finished book"
         table-line (file "~/documents/notes/books-read.org")
         "| %^{Title} | %^{Author} | %u |")

        ("r" "Reading"
         checkitem
         (file (org-file-path "to-read.org")))

        ("s" "Subscribe to an RSS feed"
         plain
         (file "~/documents/rss/urls")
         "%^{Feed URL} \"~%^{Feed name}\"")

        ("t" "Todo"
         entry
         (file+headline org-index-file "Inbox")
         "* TODO %?\n")))
\end{verbatim}

When I'm starting an Org capture template I'd like to begin in insert mode. I'm
opening it up in order to start typing something, so this skips a step.

\begin{verbatim}
(add-hook 'org-capture-mode-hook 'evil-insert-state)
\end{verbatim}

When refiling an item, I'd like to use ido for completion.

\begin{verbatim}
(setq org-refile-use-outline-path t)
(setq org-outline-path-complete-in-steps nil)
\end{verbatim}
\subsubsection{Keybindings}
\label{sec:orgbaefc77}

Bind a few handy keys.

\begin{verbatim}
(define-key global-map "\C-cl" 'org-store-link)
(define-key global-map "\C-ca" 'org-agenda)
(define-key global-map "\C-cc" 'org-capture)
\end{verbatim}

Hit \texttt{C-c i} to quickly open up my index file.

\begin{verbatim}
(defun gh/open-index-file ()
  "Open the master org TODO list."
  (interactive)
  (find-file org-index-file)
  (flycheck-mode -1)
  (end-of-buffer))

(global-set-key (kbd "C-c i") 'gh/open-index-file)
\end{verbatim}

\subsection{Sync Org-mode with Google Calendar}
\label{sec:orga81a25d}

I use \href{https://github.com/myuhe/org-gcal.el}{org-gcal} to sync my Google calendar.

\begin{verbatim}
(setq package-check-signature nil)


(use-package org-gcal
  :ensure t
  :config
(setq org-gcal-client-id "107011808994-g9s382a66p4d3f78ibkccl15sjgh7a9n.apps.googleusercontent.com"
          org-gcal-client-secret "Gjfci0moPki0d_APpcqEL3WF"
          org-gcal-file-alist '(("gmv.henriques@gmail.com" .  "~/Dropbox/org/calendar.org"))))

\end{verbatim}

I use these two hooks to sync things semi-automatically.
The first hook syncs whenever I load the agenda. Since this happens in the background, if I just added something to my calendar, I might have to reload the agenda by hitting r in the agenda view.
The second hook syncs with my Google calendar when I capture.

\begin{verbatim}
(add-hook 'org-agenda-mode-hook (lambda () (org-gcal-sync) ))
(add-hook 'org-capture-after-finalize-hook (lambda () (org-gcal-sync) ))
\end{verbatim}

\href{https://github.com/kiwanami/emacs-calfw}{Calfw} it's a nice tool to view calendars in Google.

\begin{verbatim}
  (use-package calfw
    :ensure t
    :config
    (use-package calfw-ical
    :ensure t
    :config
    (use-package calfw-org
    :ensure t
    :config
    (setq cfw:display-calendar-holidays nil)
    (defun mycalendar ()
      (interactive)
      (cfw:open-calendar-buffer
       :contents-sources
       (list
        (cfw:org-create-source "Green")
;	(cfw:ical-create-source "Gcal" "https://calendar.google.com/calendar/ical/gmv.henriques%40gmail.com/private-549e154258dff1844e9f91f62688c84b/basic.ics" "White")
        (cfw:ical-create-source "Feriados" "https://calendar.google.com/calendar/ical/pt-pt.portuguese%23holiday%40group.v.calendar.google.com/public/basic.ics" "Red")
        )))
  )
  )
  )
\end{verbatim}
\section{Editing Settings}
\label{sec:org0750d42}
\subsection{Quickly visit Emacs configuration}
\label{sec:org3f0cd8f}

I futz around with my dotfiles a lot. This binds \texttt{C-c e} to quickly open my
Emacs configuration file.

\begin{verbatim}
(defun gh/visit-emacs-config ()
  (interactive)
  (find-file "~/Dropbox/emacs/configuration.org"))

(global-set-key (kbd "C-c e") 'gh/visit-emacs-config)
\end{verbatim}

\subsection{Always kill current buffer}
\label{sec:org48a5974}

Assume that I always want to kill the current buffer when hitting \texttt{C-x k}.

\begin{verbatim}
(global-set-key (kbd "C-x k") 'gh/kill-current-buffer)
\end{verbatim}

\subsection{Use \texttt{company-mode} everywhere}
\label{sec:org3909f6b}

\begin{verbatim}
(use-package company
:ensure t
:init
(add-hook 'after-init-hook 'global-company-mode)
)
\end{verbatim}

\subsection{\texttt{Saveplace}}
\label{sec:orgbdb454b}

Purpose: When you visit a file, point goes to the last place where it was when you previously visited the same file.

\begin{verbatim}
(use-package saveplace
 :ensure t
 :init
 (save-place-mode 1)
)
\end{verbatim}

\href{https://www.emacswiki.org/emacs/SavePlace}{Save Place}
\subsection{Always indent with spaces}
\label{sec:orgfe533d7}

Never use tabs. Tabs are the devil’s whitespace.

\begin{verbatim}
(setq-default indent-tabs-mode nil)
\end{verbatim}

\subsection{Configure \texttt{yasnippet}}
\label{sec:org7c4554b}

I keep my snippets in \texttt{\textasciitilde{}/Dropbox/emacs/snippets/text-mode}, and I always want \texttt{yasnippet} enabled.

\begin{verbatim}
(use-package yasnippet
:ensure t
:init
  (setq yas-snippet-dirs '("~/Dropbox/emacs/snippets/text-mode"))
  (yas-global-mode 1))
\end{verbatim}

I don’t want \texttt{ido} to automatically indent the snippets it inserts. Sometimes this looks pretty bad (when indenting org-mode, for example, or trying to guess at the correct indentation for Python).

\begin{verbatim}
(setq yas/indent-line nil)
\end{verbatim}
\subsection{Configure \texttt{ido}}
\label{sec:org7b3134b}

\begin{verbatim}
(use-package ido
  :ensure t
  :init
  (setq ido-enable-flex-matching t)
  (setq ido-everywhere t)
  (ido-mode 1)

  (use-package flx-ido
    :ensure t
    :init
    (flx-ido-mode 1) ; better/faster matching
  )

(setq ido-create-new-buffer 'always) ; don't confirm to create new buffers

  (use-package ido-vertical-mode
    :ensure t
    :init
    (ido-vertical-mode 1)
    (setq ido-vertical-define-keys 'C-n-and-C-p-only)
  )
)
\end{verbatim}

\href{https://www.emacswiki.org/emacs/InteractivelyDoThings}{ido}
\href{https://github.com/lewang/flx}{flx-ido}
\href{https://github.com/creichert/ido-vertical-mode.el}{ido-vertical-mode}
\subsection{Electric pair}
\label{sec:orga7c8fd1}

Typing any left bracket automatically insert the right matching bracket.

\begin{verbatim}
(electric-pair-mode 1)
\end{verbatim}
\subsection{Rainbow-delimiters}
\label{sec:org7203746}

\begin{verbatim}
(use-package rainbow-delimiters
  :ensure t
  :commands rainbow-delimiters-mode
  :init
  (add-hook 'prog-mode-hook #'rainbow-delimiters-mode)
  (add-hook 'LaTex-mode-hook #'rainbow-delimiters-mode)
  (add-hook 'org-mode-hook 'rainbow-delimiters-mode))
\end{verbatim}

\subsection{Use \texttt{smex} to handle \texttt{M-x} with \texttt{ido}}
\label{sec:orgd4113d4}

\begin{verbatim}
(use-package smex
  :ensure t
  :init
  (smex-initialize)
)

(global-set-key (kbd "M-x") 'smex)
(global-set-key (kbd "M-X") 'smex-major-mode-commands)
\end{verbatim}

\subsection{Switch and rebalance windows when splitting}
\label{sec:org4e00172}

When splitting a window, I invariably want to switch to the new window. This makes that automatic.

\begin{verbatim}
 (defun gh/split-window-below-and-switch ()
  "Split the window horizontally, then switch to the new pane."
  (interactive)
  (split-window-below)
  (balance-windows)
  (other-window 1))

(defun gh/split-window-right-and-switch ()
  "Split the window vertically, then switch to the new pane."
  (interactive)
  (split-window-right)
  (balance-windows)
  (other-window 1))

(global-set-key (kbd "C-x 2") 'gh/split-window-below-and-switch)
(global-set-key (kbd "C-x 3") 'gh/split-window-right-and-switch)
\end{verbatim}
\section{Writing}
\label{sec:org2fc4c0f}
\subsection{Change dictionary}
\label{sec:orga322c04}

Change dictionary to \texttt{PT-preao}

\begin{verbatim}
(global-set-key
[f3]
(lambda ()
    (interactive)
    (ispell-change-dictionary "pt_PT-preao")))
\end{verbatim}

Change dictionary to \texttt{En}

\begin{verbatim}
(global-set-key
[f4]
(lambda ()
    (interactive)
    (ispell-change-dictionary "en")))
\end{verbatim}
\subsection{Wrap paragraphs automatically}
\label{sec:org62968ca}

\texttt{AutoFillMode} automatically wraps paragraphs.

\begin{verbatim}
(add-hook 'text-mode-hook 'turn-on-auto-fill)
(add-hook 'gfm-mode-hook 'turn-on-auto-fill)
(add-hook 'org-mode-hook 'turn-on-auto-fill)
\end{verbatim}

Sometimes, though, I don’t wanna wrap text. This toggles wrapping with \texttt{C-c q}:

\begin{verbatim}
(global-set-key (kbd "C-c q") 'auto-fill-mode)
\end{verbatim}
\subsection{\texttt{Flyspell-popup}}
\label{sec:orga1a59e5}

Call flyspell-popup-correct to correct misspelled word at point with a Popup
Menu. You might want to bind it to a short key, for example:

\begin{verbatim}
(use-package flyspell-popup
:ensure t
:init
(define-key flyspell-mode-map (kbd "C-;") #'flyspell-popup-correct))
\end{verbatim}
\subsection{Darkroom}
\label{sec:org742df63}

\begin{verbatim}
(use-package darkroom
   :ensure t)
\end{verbatim}

\section{Email with \texttt{mu4e}}
\label{sec:orgb5eb1ee}
\subsection{Evil}
\label{sec:orgb06f316}
Use the evil bindings for navigation. They’re very similar to the mutt bindings,
which matches my muscle memory nicely. =)

\begin{verbatim}
(require 'evil-mu4e)
\end{verbatim}
\subsection{Where’s my mail? Who am I?}
\label{sec:org31cec0f}
I keep my mail in \texttt{\textasciitilde{}/.mail}. The default mail directory would be \texttt{\textasciitilde{}/Maildir}, but I’d rather hide it; I don’t poke around in there manually very often.

This setting matches the paths in my mbsync configuration.

\begin{verbatim}
(setq mu4e-maildir "~/.mail")
\end{verbatim}

I only have one context at the moment. If I had another email account, though,
I’d define it in here with an additional \texttt{make-mu4e-context} block.

My full name is defined earlier in this configuration file.

\begin{verbatim}
(setq mu4e-contexts
      `(,(make-mu4e-context
          :name "gmail"
          :match-func (lambda (msg)
                        (when msg
                          (string-prefix-p "/gmail" (mu4e-message-field msg :maildir))))
          :vars '((user-mail-address . "gmv.henriques@gmail.com")
                  (mu4e-trash-folder . "/gmail/archive")
                  (mu4e-refile-folder . "/gmail/archive")
                  (mu4e-sent-folder . "/gmail/sent")
                  (mu4e-drafts-folder . "/gmail/drafts")))))
\end{verbatim}
\subsection{Fetching new mail}
\label{sec:orgc1afe38}

I fetch my email with \texttt{mbsync}. I’ve also bound “o” to fetch new mail.

\begin{verbatim}
(setq mu4e-get-mail-command "killall --quiet mbsync; mbsync inboxes")

(define-key mu4e-headers-mode-map (kbd "o") 'mu4e-update-mail-and-index)
\end{verbatim}

Rename files when moving them between directories. \texttt{mbsync} supposedly prefers
this; I’m cargo-culting.

\begin{verbatim}
(setq mu4e-change-filenames-when-moving t)
\end{verbatim}

Poll the server for new mail every 1 minute.

\begin{verbatim}
(setq mu4e-update-interval 60)
\end{verbatim}
\subsection{Viewing mail}
\label{sec:orge4f4ebb}

I check my email pretty often! Probably more than I should. This binds \texttt{C-c m}
to close any other windows and open my personal inbox.

In practice, I keep an \textbf{mu4e-headers} buffer in its own frame, full-screen, on a
dedicated i3 workspace.

\begin{verbatim}
(defun gh/visit-inbox ()
  (interactive)
  (delete-other-windows)
  (mu4e~headers-jump-to-maildir "/gmail/inbox"))

(global-set-key (kbd "C-c m") 'gh/visit-inbox)
\end{verbatim}

Open my inbox and sent messages folders with \texttt{J-i} and \texttt{J-s}, respectively.
These are the only two folders I visit regularly enough to warrant shortcuts.

\begin{verbatim}
(setq mu4e-maildir-shortcuts '(("/gmail/inbox" . ?i)
                             ("/gmail/sent" . ?s)))
\end{verbatim}

\texttt{mu4e} starts approximately instantaneously, so I don’t know why I’d want to
reconsider quitting it.

\begin{verbatim}
(setq mu4e-confirm-quit nil)
\end{verbatim}
\subsection{Composing a new message}
\label{sec:orgf9b9edb}
When I’m composing a new email, default to using the first context.

\begin{verbatim}
(setq mu4e-compose-context-policy 'pick-first)
\end{verbatim}

Compose new messages (as with \texttt{C-x m}) using \texttt{mu4e-user-agent}.

\begin{verbatim}
(setq mail-user-agent 'mu4e-user-agent)
\end{verbatim}

Enable Org-style tables and list manipulation.

\begin{verbatim}
(add-hook 'message-mode-hook 'turn-on-orgtbl)
(add-hook 'message-mode-hook 'turn-on-orgstruct++)
\end{verbatim}

Check my spelling while I'm writing.

\begin{verbatim}
(add-hook 'mu4e-compose-mode-hook 'flyspell-mode)
\end{verbatim}

Once I've sent an email, kill the associated buffer instead of just burying it.

\begin{verbatim}
(setq message-kill-buffer-on-exit t)
\end{verbatim}

\subsection{Reading an email}
\label{sec:org6a7a503}

Display the sender's email address along with their name.

\begin{verbatim}
(setq mu4e-view-show-addresses t)
\end{verbatim}

Save attachments in my \texttt{\textasciitilde{}/downloads} directory, not my home directory.

\begin{verbatim}
(setq mu4e-attachment-dir "~/downloads")
\end{verbatim}

Hit \texttt{C-c C-o} to open a URL in the browser.

\begin{verbatim}
(define-key mu4e-view-mode-map (kbd "C-c C-o") 'mu4e~view-browse-url-from-binding)
\end{verbatim}

While HTML emails are undeniably sinful, we often have to read them. That's
sometimes best done in a browser. This effectively binds \texttt{a h} to open the
current email in my default Web browser.

\begin{verbatim}
(add-to-list 'mu4e-view-actions '("html in browser" . mu4e-action-view-in-browser) t)
\end{verbatim}

\subsection{Encryption}
\label{sec:org39dd0a2}

If a message is encrypted, my reply should always be encrypted, too.

\begin{verbatim}
(defun gh/encrypt-responses ()
  (let ((msg mu4e-compose-parent-message))
    (when msg
      (when (member 'encrypted (mu4e-message-field msg :flags))
        (mml-secure-message-encrypt-pgpmime)))))

(add-hook 'mu4e-compose-mode-hook 'gh/encrypt-responses)
\end{verbatim}

\subsection{Sending mail over SMTP}
\label{sec:org3b482ad}

I send my email through \texttt{msmtp}. It's very fast, and I've already got it
configured from using \texttt{mutt}. These settings describe how to send a message:

\begin{itemize}
\item Use a sendmail program instead of sending directly from Emacs,
\item Tell \texttt{msmtp} to infer the correct account from the \texttt{From:} address,
\item Don't add a "\texttt{-f username}" flag to the \texttt{msmtp} command, and
\item Use \texttt{/usr/bin/msmtp}!
\end{itemize}

\begin{verbatim}
(setq message-send-mail-function 'message-send-mail-with-sendmail)
(setq message-sendmail-extra-arguments '("--read-envelope-from"))
(setq message-sendmail-f-is-evil 't)
(setq sendmail-program "msmtp")
\end{verbatim}

\subsection{Org integration}
\label{sec:org2515b6d}

\texttt{org-mu4e} lets me store links to emails. I use this to reference emails in my
TODO list while keeping my inbox empty.

\begin{verbatim}
(require 'org-mu4e)
\end{verbatim}

When storing a link to a message in the headers view, link to the message
instead of the search that resulted in that view.

\begin{verbatim}
(setq org-mu4e-link-query-in-headers-mode nil)
\end{verbatim}
\subsection{Configure BBDB with mu4e}
\label{sec:org8dc4089}

Use BBDB to handle my address book.

\begin{verbatim}
(require 'bbdb-mu4e)
\end{verbatim}

Don’t try to do address completion with mu4e. Use BBDB instead:

\begin{verbatim}
(setq mu4e-compose-complete-addresses nil)
\end{verbatim}
\subsection{Try to display images in mu4e}
\label{sec:org043c99e}

\begin{verbatim}
(setq
 mu4e-view-show-images t
 mu4e-view-image-max-width 800)
\end{verbatim}
\section{My \texttt{latex} environment}
\label{sec:org44e6f80}
\subsection{\texttt{Auctex}}
\label{sec:org5f25adf}


\begin{verbatim}

(use-package tex-site
  :ensure auctex
  :mode ("\\.tex\\'" . latex-mode)
  :config
  (setq TeX-auto-save t)
  (setq TeX-parse-self t)
  (setq-default TeX-master nil)
  (add-hook 'LaTeX-mode-hook
            (lambda ()
              (rainbow-delimiters-mode)
              (company-mode)
              (smartparens-mode)
              (turn-on-reftex)
              (setq reftex-plug-into-AUCTeX t)
              (reftex-isearch-minor-mode)
              (setq TeX-PDF-mode t)
              (setq TeX-source-correlate-method 'synctex)
              (setq TeX-source-correlate-start-server t)))

;; Update PDF buffers after successful LaTeX runs
(add-hook 'TeX-after-TeX-LaTeX-command-finished-functions
           #'TeX-revert-document-buffer)

;; to use pdfview with auctex
(add-hook 'LaTeX-mode-hook 'pdf-tools-install)

;; to use pdfview with auctex
(setq TeX-view-program-selection '((output-pdf "pdf-tools"))
       TeX-source-correlate-start-server t)
(setq TeX-view-program-list '(("pdf-tools" "TeX-pdf-tools-sync-view"))))
\end{verbatim}

\subsection{\texttt{Reftex}}
\label{sec:org369e28d}

\begin{verbatim}
(use-package reftex
  :ensure t
  :defer t
  :config
  (setq reftex-cite-prompt-optional-args t)); Prompt for empty optional arguments in cite
\end{verbatim}
\subsection{\texttt{Ivy-bibtex}}
\label{sec:org39acbe7}
\subsection{\texttt{Pdf-tools}}
\label{sec:orgd4c33dc}
\begin{verbatim}
(use-package pdf-tools
  :ensure t
  :mode ("\\.pdf\\'" . pdf-tools-install)
  :bind ("C-c C-g" . pdf-sync-forward-search)
;  :defer t
  :config
  (setq mouse-wheel-follow-mouse t)
  (setq pdf-view-resize-factor 1.10))
\end{verbatim}
\subsection{\texttt{magic-latex-buffer}}
\label{sec:org4f61b43}

\begin{verbatim}
(use-package magic-latex-buffer
  :ensure t)
(add-hook 'LaTeX-mode-hook 'magic-latex-buffer)
\end{verbatim}
\section{\texttt{Elfeed}}
\label{sec:org70b271b}

\begin{verbatim}
(use-package elfeed-org
  :ensure t
  :config
  (progn
    (elfeed-org)
    (setq rmh-elfeed-org-files (list "~/Dropbox/org/feeds.org"))))
\end{verbatim}

\begin{verbatim}
(use-package elfeed)
  :ensure t
\end{verbatim}
\section{Header}
\label{sec:orgbe33a77}

Time-stamp!

\begin{verbatim}

(add-hook 'before-save-hook 'time-stamp)

(setq
  time-stamp-pattern nil
  time-stamp-active t          ; do enable time-stamps
  time-stamp-line-limit 12     ; check first 10 buffer lines for Time-stamp:
  time-stamp-format "%04y-%02m-%02d %02H:%02M:%02S (%u)") ; date format
\end{verbatim}

\texttt{Header2.el}!

\begin{verbatim}

(use-package header2
  :config
  (progn

    (defconst gh/header-sep-line-char ?-
      "Character to be used for creating separator lines in header.")

    (defconst gh/header-border-line-char ?=
      "Character to be used for creating border lines in header.")

    (defconst gh/auto-headers-hooks '(latex-mode-hook
                                      LaTeX-mode-hook)
      "List of hooks of major modes in which headers should be auto-inserted.")

    (defvar gh/header-timestamp-cond (lambda () t)
      "This variable should be set to a function that returns a non-nil
      value only when the time stamp is supposed to be inserted. By default, it's
      a `lambda' return `t', so the time stamp is always inserted.")

    (defun gh/turn-on-auto-headers ()
      "Turn on auto headers only for specific modes."
      (interactive)
      (dolist (hook gh/auto-headers-hooks)
        (add-hook hook #'auto-make-header)))

    (defun gh/turn-off-auto-headers ()
      "Turn off auto headers only for specific modes."
      (interactive)
      (dolist (hook gh/auto-headers-hooks)
        (remove-hook hook #'auto-make-header)))


    (defsubst gh/header-sep-line ()
      "Insert separator line"
      (insert header-prefix-string)
      (insert-char gh/header-sep-line-char (- fill-column (current-column)))
      (insert "\n"))

    (defsubst gh/header-border-line ()
      "Insert separator line"
      (insert header-prefix-string)
      (insert-char gh/header-border-line-char (- fill-column (current-column)))
      (insert "\n"))


    (defsubst gh/header-file-name ()
      "Insert \"File Name\" line, using buffer's file name."
      (insert header-prefix-string "File Name          : "
              (if (buffer-file-name)
                  (file-name-nondirectory (buffer-file-name))
                (buffer-name))
              "\n"))

    (defsubst gh/header-author ()
      "Insert current user's name (`user-full-name') as this file's author."
      (insert header-prefix-string "Author             : "
              (user-full-name)
              "\n"))

    (defsubst gh/header-mail ()
      "Insert current user's name (`user-mail-address') as this file's author."
      (insert header-prefix-string "Author e-mail      : "
              user-mail-address
              "\n"))

    (defsubst gh/header-description ()
      "Insert \"Description\" line."
      (insert header-prefix-string "Description        : \n"))

    (defsubst gh/header-creation-date ()
      "Insert todays date as the time of last modification."
      (insert header-prefix-string "Created            : "
              (header-date-string)
              "\n"))

    (defsubst gh/header-timestamp ()
      "Insert field for time stamp."
      (when (funcall gh/header-timestamp-cond)
      (insert header-prefix-string "Time-stamp: <>\n")))

    (defsubst gh/header-modification-date ()
      "Insert todays date as the time of last modification.
       This is normally overwritten with each file save."
      (insert header-prefix-string "Last-Updated       :"
              "\n"))


    (defsubst gh/header-position-point ()
      "Position the point at a particular point in the file.
Bring the point 2 lines below the current point."
      (forward-line 0)
      (newline 2))


    (setq make-header-hook '(gh/header-border-line
                             header-blank
                             gh/header-file-name
                             gh/header-author
                             gh/header-mail
                             gh/header-creation-date
                             header-blank
                             gh/header-sep-line
                             header-blank
                             gh/header-timestamp
                             header-blank
                             gh/header-sep-line
                             header-blank
                             gh/header-description
                             header-modification-date
                             header-blank
                             gh/header-border-line
                             gh/header-position-point))
    (gh/turn-on-auto-headers)
    ))

\end{verbatim}

\section{\texttt{Dired}}
\label{sec:org0cdb77a}
Load up the assorted \texttt{dired} extensions.

\begin{verbatim}
(use-package dired-details)
(use-package dired+)
\end{verbatim}

Open media with the appropriate programs.

\begin{verbatim}
(use-package dired-open
  :config
  (setq dired-open-extensions
        '(("pdf" . "evince")
          ("mkv" . "vlc")
          ("mp4" . "vlc")
          ("avi" . "vlc"))))
\end{verbatim}

These are the switches that get passed to \texttt{ls} when \texttt{dired} gets a list of
files. We’re using:

\begin{itemize}
\item \texttt{l}: Use the long listing format.
\item \texttt{h}: Use human-readable sizes.
\item \texttt{v}: Sort numbers naturally.
\item \texttt{A}: Almost all. Doesn't include "\texttt{.}" or "\texttt{..}".
\end{itemize}

\begin{verbatim}
(setq-default dired-listing-switches "-lhvA")
\end{verbatim}

Use "j" and "k" to move around in \texttt{dired}.

\begin{verbatim}
(evil-define-key 'normal dired-mode-map (kbd "j") 'dired-next-line)
(evil-define-key 'normal dired-mode-map (kbd "k") 'dired-previous-line)
\end{verbatim}

Kill buffers of files/directories that are deleted in \texttt{dired}.

\begin{verbatim}
(setq dired-clean-up-buffers-too t)
\end{verbatim}

Always copy directories recursively instead of asking every time.

\begin{verbatim}
(setq dired-recursive-copies 'always)
\end{verbatim}

Ask before recursively \emph{deleting} a directory, though.

\begin{verbatim}
(setq dired-recursive-deletes 'top)
\end{verbatim}

Open a file with an external program (that is, through \texttt{xdg-open}) by hitting
\texttt{C-c C-o}.

\begin{verbatim}
(defun dired-xdg-open ()
  "In dired, open the file named on this line."
  (interactive)
  (let* ((file (dired-get-filename nil t)))
    (call-process "xdg-open" nil 0 nil file)))

(define-key dired-mode-map (kbd "C-c C-o") 'dired-xdg-open)
\end{verbatim}
\section{Projectile}
\label{sec:org8378911}

\begin{verbatim}
(use-package projectile
  :ensure t
  :config
  (projectile-global-mode))
\end{verbatim}
\section{Hydra}
\label{sec:org9d67f6a}

\begin{verbatim}
(use-package hydra
  :ensure t)
\end{verbatim}
\section{Engine-mode}
\label{sec:org11ac996}

Enable engine-mode and define a few useful engines.

\begin{verbatim}
(use-package engine-mode
 :ensure t
 :config
 (engine/set-keymap-prefix (kbd "C-s"))
 (progn
    (defengine priberam
    "https://www.priberam.pt/dlpo/%s"
    :keybinding "p")

 (engine-mode t)))
\end{verbatim}
\end{document}
